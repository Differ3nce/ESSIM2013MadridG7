\documentclass{beamer}


\usepackage{graphicx}
\usepackage{parskip}

\usetheme{Madrid}

\AtBeginSection[]
{
  \begin{frame}
    \frametitle{Table of Contents}
    \tableofcontents[currentsection]
  \end{frame}
}


\title{Modelling the Trajectory of a Skydiver}
\author{}
\institute{}

\begin{document}

\begin{frame}
\titlepage
\end{frame}


\section{Introduction}

\begin{frame}
\frametitle{Skydiving}
\begin{columns}[c]
\column{3in}
\begin{itemize}
\item Risk during skydiving should
be minimized
\item Cybernetic Parachute Release
System (CYPRES)
\item Determines height from
measured barometric
pressure
\item Between 1991 and 2003 over
1000 fatal accidents
prevented
\item Many limitations, needs improvement
\end{itemize}
\column{2in}
\includegraphics[width=40mm,height=60mm]{skydivers.eps}
\end{columns}
\end{frame}

\begin{frame}
\frametitle{Considerations}
\begin{columns}[c]
\column{3in}
\begin{itemize}
\item Aim :  Model the trajectory of a skydiver as he falls
\item Important Factors : Wind, starting velocity, drag due to the diver's posture and positioning during the skydive
\item Difficulties : Measured data is very noisy due to changes in position and posture in the individual diver
\item Goal : Data can be calibrated onto the physical model and the noise can be analysed and cleaned up.
\end{itemize}
\column{2in}
\includegraphics[width=50mm,height=50mm]{KK_skydive.eps}
\end{columns}
\end{frame}


\section{Filtering}

\begin{frame}
\frametitle{Jump Points}
\begin{columns}[c]
\column{2in}
\begin{itemize}
\item Access to the data for acceleration in x,y and z axis and the height measurement of the diver
\item Want to predict when the diver jumps
\item This point can be measured if we filter the acceleration data
\item Median filter used with window of size $21$ timesteps (Delay$=2.5s$)
\end{itemize}
\column{3in}
\includegraphics[width=80mm,height=60mm]{filtered_data.eps}
\end{columns}
\end{frame}

\begin{frame}
\frametitle{Initial Velocity Approximation}
\begin{columns}[c]
\column{2.5in}
\begin{itemize}
\item We need an approximation of the initial velocity to create an ODE model
\item We do this by filtering the measured height data and then differentiating
\item Moving Average filter applied
\item Delay taken due to pressure change caused by the opening of the hatch
\end{itemize}
\column{2.5in}
\includegraphics[width=70mm,height=60mm]{filtered_height.eps}
\end{columns}
\end{frame}

\begin{frame}
\frametitle{Physics}
\begin{itemize}
\item Barometric formulas\\ 
	Pressure
	\begin{equation*}
	\frac{P}{P_0} = \left(\frac{T_0}{T_0+L_0h}\right)^{\frac{gM}{R^*L_0}}
	\end{equation*}
	Air Density
	\begin{equation*}
	\rho = \rho _0 \left( \frac{T_0+L_0 h}{T_0}\right) ^{\left(-\frac{gM}{R^*L_0}\right)-1}
	\end{equation*}
	
\item Wind drag
	\begin{equation*}
	F_D=\frac{1}{2}\rho v^2c_DA
	\end{equation*}
\item Gravitational acceleration $g$ can be kept constant.
\end{itemize}
\end{frame}

\begin{frame}
\frametitle{ODE}
\begin{align*}
&F=m\cdot a \quad\text{(Newtons's second law)}\\
&\frac{d^2z}{dt^2}=\frac{1}{m}F_D-g\\
\end{align*}
\begin{align*}
&\frac{d^2z}{dt^2}=\frac{1}{2m}\rho(z)\left(\frac{dz}{dt}\right)^2\underbrace{c_{D}A}_{=:c^*(t)}-g
\end{align*}
$\rho(z) = \rho _0 \left( \frac{T_0+L_0 z}{T_0}\right) ^{(-\frac{gM}{R^*L_0})-1}$

Difficulties:\\
Drag coefficient $c_{D}$\\ 
Cross section area $A$ depend on posture of diver
\end{frame}

\begin{frame}
\frametitle{Approximation of the parameter $c^*$}
Method: Non-linear data fitting for ODEs

Given: Data points ${(t_i,y_i)}_{i=1}^m$

Estimate parameter $c^*$ s.t.
\begin{align*}
\underset{c\in\mathbb{R}}{\min}\quad &\phi=\frac{1}{2}\sum_{i=1}^m\Vert\hat{y}(t_i)-y_i\Vert_2^2\\
s.t.\quad &\frac{d\Bx}{dt}(t)=f(t,\Bx(t),c)\quad \Bx(t_0)=\Bx_0\\
	&\hat{y}(t)=g(\Bx(t),c)\\
	&c_l\leq c\leq c_u,
\end{align*}

\end{frame}


\end{document}
