%When skydiving there can be situations when

In recent years, the extreme sport of skydiving has become increasingly popular. This, combined with the development 
and availability of cheap accelerometers has generated a market for Automatic Activation Devices (AAD). AAD's 
automatically release a reserve parachute when they detect the diver's velocity is greater than a predetermined
value at a preset altitude. One such device is the CYPRES (Cybernetic Parachute Release System) produced by Airtec.
The CYPRES is availble in four different models -  Expert, Student, Tandem and Speed. The devices differ only in the 
activation parameters.

In the trajectory of a skydiver there is a critical point where he should switch on his parachute. 
After this point consequences can be disastrous if he hasn't switched it on yet. 
Our aim is to predict the skydiver's height in order to the parachute switches on automatically when he is too near of 
the floor.\\
%improve his security.\\
We design a model based on free fall law and wind drag force. 
Furthermore, we have some data about six different skydivers. 
The idea is to use our ODE model and the available data to predict the skydiver's height. 
