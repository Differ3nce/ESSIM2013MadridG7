%\documentclass[11pt]{article}

% Do not modify the next 7 lines.
\topmargin -0.5in
\footskip 0.7in
\textwidth 6.5in
\textheight 9.0in
\oddsidemargin 0.1in
\evensidemargin 0.1in
\parindent0pt\parskip1ex



%\begin{document}

In recent years, the extreme sport of skydiving has become increasingly popular. This, combined with the development
and availability of cheap accelerometers has generated a market for Automatic Activation Devices (AADs). AADs
automatically release a reserve parachute when they detect that the diver's velocity is greater than a predetermined
value at a preset altitude. One such device is the CYPRES (Cybernetic Parachute Release System) produced by Airtec.
Ideally, the creators of CYPRES would like these emergency instruments to determine the exact height of the diver above the ground by using the already recorded data, and then, in case of emergencies, automatically deploy the parachute. However, due to the changing position and posture of the diver, the measured data
is very noisy. The objective of this project was to develop a physical model of the skydiver and use it to calibrate the measured data, reducing the amount of noise present in the signal. An ODE model has been developed based on the free fall and drag of the diver. The initial conditions for the model have been taken from the measured data after some filtering. An extended, non-linear, Kalman filter was then applied to give a reasonable prediction of the diver's height based on previously recorded data.

%\end{document}