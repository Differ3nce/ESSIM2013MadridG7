\documentclass[12pt,a4paper]{article}
\usepackage[utf8]{inputenc}
\usepackage[english]{babel}
\usepackage{amsmath}
\usepackage{amsfonts}
\usepackage{amssymb}
\usepackage{graphicx}
\frenchspacing
\usepackage{parskip}
\author{Group 7}

%SOME BOLD LETTERS
\newcommand{\Bx}{\textbf{x}}
\newcommand{\Bu}{\textbf{u}}
\newcommand{\BP}{\textbf{P}}
\newcommand{\BF}{\textbf{F}}
\newcommand{\BC}{\textbf{C}}
\newcommand{\BK}{\textbf{K}}
\newcommand{\BS}{\textbf{S}}
\newcommand{\By}{\textbf{y}}
\newcommand{\Bz}{\textbf{z}}
\newcommand{\Bw}{\textbf{w}}
\newcommand{\Bv}{\textbf{v}}
%SOME MATH

\newcommand{\zVec}
{
	\left[
	\begin{matrix}
		h_{k} \\
		\dot{\| v_{k} \|}
	\end{matrix}\right]
}

\newcommand{\xVec}
{
	\left[
	\begin{matrix}
		h_{k} \\
		\| v_{k} \|
		\end{matrix}\right]
}

\newcommand{\unknownH}
{
	\left[
	\begin{array}{cc}
		c_{11} & c_{12} \\
		c_{21} & c_{22}
	\end{array}
	\right]
}

\newcommand{\cField}
{
	\left\lbrace
	\begin{array}{l}
	h \\ 
	0.5\rho_0\left(\frac{T_0+L_0h}{T_0}\right)^{\frac{-g_0M} {R^*L_0}-1}
	\end{array} 
	\right.
}

\begin{document}
\title{Skydiver}
\maketitle

\begin{center}
\textbf{Miriam Ruiz Ferrández}\\
\textit{Faculty of Mathematics, University of Santiago de Compostela,\\
Galicia, Spain}

\textbf{Philipp M\"{u}ller}\\
\textit{Faculty of Engineering Sciences, Tampere University of Technology,\\
P.O. Box 692, FI-33101 Tampere, Finland}

\textbf{Katharina Rafetseder}\\
\textit{Faculty of Mathematics, Johannes Kepler Universit\"{a}t Linz,\\
Altenberger Str. 69, 4040 Linz, Austria}

\textbf{Tom Slenders}\\
\textit{Faculty of Mathematics and Computer Science, Eindhoven University of Technology,\\
Den Dolech 2, Postbus 513, 5600 MB Eindhoven, The Netherlands}

\textbf{Angelos Toytziaridis}\\
\textit{Faculty of Engineering, University of Lund,\\
Box 117, SE-221 00  Lund, Sweden}\\
\vspace{1cm}
\textbf{Instructor: K. Kulshreshtha }\\
\textit{ ,University of Paderborn,\\
, , Germany}

\end{center}

\thispagestyle{empty}
\newpage
\begin{abstract}
Modelling the trajectory of a skydiver.
\end{abstract}
\newpage
\section{Introduction}
%When skydiving there can be situations when
In the trajectory of a skydiver there is a critical point where he should switch on his parachute. After this point consequences can be disastrous if he hasn't switched it on yet. Our aim is to predict the skydiver's height in order to the parachute switches on automatically when he is too near of the floor.\\
%improve his security.\\
We design a model based on free fall law and wind drag force. Furthermore, we have some data about six different skydivers. The idea is to use our ODE model and the available data to predict the skydiver's height. 

\section{Problem formulation}
During the skydiving a small device measures the following data: pressure and acceleration (using triaxial accelerometers). This measurements are done every $0.25$\,s. The goal is now to use the last few of the measurements to determine the skydiver's height and velocity during his jump. The difficulty is that the height as well as the acceleration includes a lot of noise due to the skydiver rotating and changing position all the time. In case of the pressure the measurement errors are a result from the acting wind. In a situation as in figure \ref{fig:pressure_c} and figure \ref{fig:pressure_d} there will be no influence whereas in figure \ref{fig:pressure_a} and figure \ref{fig:pressure_b} we will measure more and less pressure, respectively. Since the height is calculated from the pressure we always have to take noise into account for our calculations. In spite of this the height measurements can be expected to be more accurate than the acceleration measured by the accelerometer. Moreover, the computation should be low on resources because it should run on a small device.
\input{Pictures/Fig_pressure_total}
The idea is to deploy the parachute in case the height is smaller than some critical height and the velocity is still higher than the critical velocity, which means the parachute has not been deployed by the skydiver on his own. A list of the critical heights and velocities is listed below:\\
\hfill\\
\begin{tabular}{|l|r|r|r|r|}
\hline
     Type of diver & critical height  & critical velocity\\ \hline
	Experts &  225\,$m$ & 35\,$\frac{m}{s}$  \\ \hline
	Students & 300\,$m$  & 13\,$\frac{m}{s}$ \\ \hline
	Tandem&  580\,$m$  & 35\,$\frac{m}{s}$  \\ \hline
	Speed&  100-225\,$m$  & 43\,$\frac{m}{s}$  \\ \hline
\end{tabular} 


\section{Approach}
\subsection*{Method 1}
\begin{enumerate}
	\item Create an ODE.
	\item Estimate unknown parameters.
	\item Create a extended Kalman filter.
\end{enumerate}
Our idea is to
\subsection*{Method 2}
\begin{enumerate}
	\item Create an ODE.
	\item Estimate unknown parameters.
	\item Create a ARMA model.
\end{enumerate}

\section{Theory}
\subsection{Given equations}
We let $h$ be the height and $v$ the velocity in the z-direction.
Barometric formula, from which we get the velocity.
\begin{equation}
\frac{P}{P_0} = \left(\frac{T_0}{T_0+L_0h}\right)^{\frac{gM}{R^*L_0}}
\label{eq:BarometricFormula}
\end{equation}
Wind drag
\begin{equation}
D=\frac{1}{2}\varrho v^2c_DA
\label{eq:WindDrag}
\end{equation}
Air density
\begin{equation}
\rho = \rho _0 \left( \frac{T_0+L_0 h}{T_0}\right) ^{\left(-\frac{gM}{R^*L_0}\right)-1}
\label{eq:AirDensity}
\end{equation}
Where the standard Temperature lapse rate $L_0=-0.0065\,\frac{K}{m}$, standard Temperature $T_0=293\,K$, density $\rho_0=1.2041$, Molar mass of air $M=0.0289644\,\frac{kg}{mol}$ and the Universal gas constant $R^*=8.31432$. 

\subsection{Assumptions}
The gravitational acceleration can be kept constant because there is only a notable change every 100 km.

\subsection{ODE}
We find that the z-acceleration is $\frac{d^2z}{dt^2} = \frac{D}{m} - g$, where $D$ is the wind drag \eqref{eq:WindDrag}, $g=9.81$ m$/$s$^2$ is the gravitational acceleration and $m$ is the mass of the diver.
After inserting we receive the following equation
\begin{align}
\frac{d^2z}{dt^2}=\frac{1}{2m}\rho(z)\left(\frac{dz}{dt}\right)^2\underbrace{c_{D}A}_{=:c^*(t)}-g,
\end{align}
with $\rho(z) = \rho _0 \left( \frac{T_0+L_0 z}{T_0}\right) ^{(-\frac{gM}{R^*L_0})-1}$.
In the following we will assume that the parameter $c^*$ is constant over the time.

\subsection{Approximation of parameters}
The next step is to approximate the unknown parameter $c^*:=c_{D}A$, which depends for example on the way the diver dives. Hence, the best way is to determine this parameter by means of non-linear data fitting for ODEs. The first step is to rewrite the equation to the following system of equations
\begin{align*}
\frac{dz}{dt}&=v(t)\\
\frac{dv}{dt}&=\frac{1}{2m}\rho(z)v(t)^2c^*-g.
\end{align*}
Given the data points ${(t_i,y_i)}_{i=1}^m$ we compute the estimate of the parameter $c^*$ such that
\begin{align}
\underset{c\in\mathbb{R}}{\min}\quad &\phi=\frac{1}{2}\sum_{i=1}^m\Vert\hat{y}(t_i)-y_i\Vert_2^2\\
s.t.\quad &\frac{d\Bx}{dt}(t)=f(t,\Bx(t),c)\quad \Bx(t_0)=\Bx_0\\
	&\hat{y}(t)=g(\Bx(t),c)\\
	&c_l\leq c\leq c_u,
\end{align}
where in our case
\begin{align*}
\Bx(t)=\left[
\begin{array}{c}
z(t)\\
v(t)\\
\end{array}
\right],
\end{align*}

\begin{align*}
f(t,x(t),c)=\left[
\begin{array}{c}
v(t)\\
\frac{1}{2m}\rho(z)v(t)^2c(t)-g\\
\end{array}
\right],
\end{align*}

\begin{align*}
g(\Bx(t),c)=z(t).
\end{align*}

One should note that we need some initial values for the height $z(t_0)$ and the velocity $v(t_0)$.


\subsection{Extended Kalman filter}
HELLO
We begin with writing our ODE as a system of first order ODE:s.
\begin{equation*}
	\frac{d^2h}{dt^2} = \frac{D}{m} - g \Longleftrightarrow f(h,v)=
	\left\lbrace
	\begin{array}{ll}
		\dot{h}= v & , h(t_0)=h_0 \\
		\dot{v}= \frac{1}{2m}\varrho v(t)^2c_DA-g& , v(t_0)=v_0
	\end{array}
	\right.
\end{equation*}
The extended Kalman filter can be formulated as
\begin{equation*}
	\begin{array}{l}
%		\Bx_k = \xVec= f(\Bx_{k-1})+\Bw_{k-1}=\BF\Bx_{k-1}+\Bw_{k-1}\\
%		\Bz_k = \zVec=  c(\Bx_k)+\Bv_k = \BC\Bx_k+\Bv_k
		\Bx_k = \xVec= f(\Bx_{k-1})+\Bw_{k-1}\\
		\Bz_k = \zVec=  c(\Bx_k)+\Bv_k
	\end{array}
\end{equation*}
where $c(h,v) = \cField$. \\
Further on both $\Bv$ and $\Bw$ are the state transition and observation noises which are both assumed to be zero mean multivariate Gaussian noises with covariance $\textbf{Q}_k$ and $\textbf{R}_k$ respectively. In our case all of these are assumed to be zero.
The process of applying the extended Kalman filter can divided into two parts predict and update.\\
\subsubsection*{Predict}
\begin{tabbing}
\hspace{6cm}\=\kill
Predicted state estimate \> $\hat{\Bx}_{k|k-1}=f(\hat{\Bx}_{k-1|k-1})$ \\
Predicted covariance estimate\>$\BP_{k|k-1} = \BF_{k-1}\BP_{k-1|k-1}\BF_{k-1}^T+\textbf{Q}_{k-1}$
\end{tabbing}
\subsubsection*{Update}
\begin{tabbing}
\hspace{8cm}\=\kill
Innovation or measurement residual\> $\tilde{\By}= \Bz_k-c(\hat{\Bx}_{k|k-1})$\\
Innovation (or residual) covariance\> $\BS_k=\BC_k\BP_{k|k-1}\BC^T_k+\textbf{R}_k$\\
Near-optimal Kalman gain\> $\BK=\BP_{k|k-1}\BC^T_k\BS^{-1}_k$\\
Updated state estimate\> $\hat{\Bx}_{k|k}=\hat{\Bx}_{k|k-1}+\BK_k\tilde{\By}_k$\\
Updated estimate covariance\> $\BP_{k|k}=(\textbf{I}-\BK_k\BC_k)\BP_{k|k-1}$
\end{tabbing}
where the state transition matrix are defined to be the following Jacobians
\begin{equation*}
	\BF_{k-1}=\left. \frac{\partial f}{\partial \Bx}\right|_{{\hat{\Bx}_{k-1|k-1},			\Bu_{k-1}}}
\end{equation*}
\begin{equation*}
	\BC_{k}=\left. \frac{\partial c}{\partial \Bx}\right|_{{\hat{\Bx}_{k|k-1},			\Bu_{k-1}}}
\end{equation*}


%\section{ARIMA model}
%\input{arima}

\section{Result}
\input{result}

\section{Conclusion}
\input{conclusion}

\bibliography{sources}{}
\bibliographystyle{plain}

\end{document}
